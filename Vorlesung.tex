\documentclass[aspectratio=169]{beamer}
% For more themes, color themes and font themes, see:
% http://deic.uab.es/~iblanes/beamer_gallery/index_by_theme.html

\mode<presentation>
{
  \usetheme{Pittsburgh} 
  \usecolortheme{beaver} % or try albatross, beaver, crane, ...
  \usefonttheme{default}  % or try serif, structurebold, ...
  \setbeamertemplate{navigation symbols}{}
  \setbeamertemplate{caption}[numbered]
} 

\usepackage[english]{babel}
\usepackage[utf8x]{inputenc}
\usepackage{eurosym}


\title[Your Short Title]{Software-Engineering}
\author{Thomas Kindermann}
\institute{HSRW}
\date{Sommersemester 2019}

\begin{document}

\begin{frame}
  \titlepage
\end{frame}

\begin{frame}{Inhalt}
  \tableofcontents
\end{frame}

\section{Einführung}
\begin{frame}{Einführung (0)}
\includegraphics[scale=0.25]{pexels-photo_development.jpg}
\end{frame}

\begin{frame}{Einführung (1)}
\begin{itemize}
  \item Der deutschsprachige Begriff “Softwaretechnik” ist gleichbedeutend mit “Software-Engineering”.

“Software-Engineering ist eine technische Disziplin, die sich mit allen Aspekten der Softwareherstellung beschäftigt, von der frühen Phase der Systemspezifikation bis hin zur Wartung des Systems, nachdem der Betrieb aufgenommen wurde."

Ian Sommerville 2012, ISBN 978-3-86894-099-2
\end{itemize}
\end{frame}
\end{document}